\documentclass{article}
\usepackage{amsmath, amssymb, graphicx}
\usepackage{geometry}
\geometry{a4paper, margin=1in}

\title{Mathematical Foundation for Backbone-Based Extraction of Duffing Nonlinearity}
\author{The $\alpha$-Catalog Initiative}
\date{\today}

\begin{document}

\maketitle

\begin{abstract}
This document provides the rigorous mathematical derivation supporting the ``Hidden Ridge'' method for extracting the cubic nonlinearity coefficient $\alpha$ from free-decay acoustic signals. We employ the Method of Averaging (Slow-Flow Approximation) on the damped Duffing equation to derive the amplitude-dependent frequency relationship (the ``Backbone Curve'') and demonstrate its independence from linear damping to leading order.
\end{abstract}

\section{The Duffing Oscillator Model}

We consider a single-degree-of-freedom oscillator with linear viscous damping and a cubic stiffness nonlinearity (Duffing oscillator). The equation of motion is given by:
\begin{equation}
    \ddot{x} + 2\gamma\dot{x} + \omega_0^2 x + \alpha x^3 = 0
    \label{eq:duffing}
\end{equation}
where:
\begin{itemize}
    \item $x(t)$ is the displacement (or modal amplitude).
    \item $\gamma$ is the viscous damping coefficient ($\gamma = \zeta \omega_0$).
    \item $\omega_0$ is the linear natural angular frequency.
    \item $\alpha$ is the cubic nonlinearity coefficient. $\alpha > 0$ implies a hardening spring; $\alpha < 0$ implies softening.
\end{itemize}

\section{The Slow-Flow Approximation}

We assume the system is weakly nonlinear ($|\alpha x^3| \ll \omega_0^2 |x|$) and lightly damped ($\gamma \ll \omega_0$). Under these conditions, the solution can be approximated as a modulated oscillation with slowly varying amplitude $a(t)$ and phase $\phi(t)$:
\begin{equation}
    x(t) = a(t) \cos(\omega_0 t + \beta(t)) = a(t) \cos \psi(t)
    \label{eq:ansatz}
\end{equation}
where $\psi(t) = \omega_0 t + \beta(t)$. The instantaneous frequency is $\omega(t) = \dot{\psi}(t) = \omega_0 + \dot{\beta}(t)$.

We impose the constraint that the amplitude and phase vary slowly compared to the oscillation frequency:
\begin{equation}
    \dot{x}(t) = -a(t) \omega_0 \sin \psi(t)
\end{equation}
This implies the constraint (from the method of variation of parameters):
\begin{equation}
    \dot{a} \cos \psi - a \dot{\beta} \sin \psi = 0
    \label{eq:constraint}
\end{equation}

Differentiating $\dot{x}(t)$:
\begin{equation}
    \ddot{x} = -\dot{a}\omega_0 \sin \psi - a\omega_0(\omega_0 + \dot{\beta}) \cos \psi
\end{equation}

Substituting $x, \dot{x}, \ddot{x}$ into Eq. (\ref{eq:duffing}):
\begin{equation}
    -\dot{a}\omega_0 \sin \psi - a\omega_0^2 \cos \psi - a\omega_0 \dot{\beta} \cos \psi - 2\gamma a \omega_0 \sin \psi + \omega_0^2 a \cos \psi + \alpha a^3 \cos^3 \psi = 0
\end{equation}

Simplifying and using the identity $\cos^3 \psi = \frac{3}{4}\cos \psi + \frac{1}{4}\cos 3\psi$:
\begin{equation}
    -\omega_0 \dot{a} \sin \psi - a\omega_0 \dot{\beta} \cos \psi - 2\gamma a \omega_0 \sin \psi + \alpha a^3 \left( \frac{3}{4}\cos \psi + \frac{1}{4}\cos 3\psi \right) = 0
    \label{eq:substituted}
\end{equation}

\section{The Method of Averaging}

We resolve Eq. (\ref{eq:substituted}) and Eq. (\ref{eq:constraint}) for $\dot{a}$ and $\dot{\beta}$. Multiplying Eq. (\ref{eq:constraint}) by $\omega_0 \cos \psi$ and Eq. (\ref{eq:substituted}) by $-\sin \psi$ (and other combinations) allows us to isolate the derivatives. However, the Method of Averaging states that for variables changing slowly over one period $T = 2\pi/\omega_0$, we can average the equations over a cycle. Fast oscillating terms (like $\sin \psi \cos \psi$, $\cos 3\psi \sin \psi$) average to zero.

\subsection{Amplitude Equation}
The averaged equation for amplitude is dominated by the damping term:
\begin{equation}
    \dot{a} = -\gamma a
\end{equation}
This yields the exponential decay $a(t) = a_0 e^{-\gamma t}$, consistent with linear theory to first order.

\subsection{Frequency Equation}
The averaged equation for phase rate $\dot{\beta}$ involves the cosine terms. The $\alpha x^3$ term contributes via $\frac{3}{4}\alpha a^3 \cos \psi$.
\begin{equation}
    -2\omega_0 \dot{\beta} a + \frac{3}{4}\alpha a^3 = 0
\end{equation}
This result is derived by substituting the approximate solution into the governing equation and enforcing the condition that secular terms (terms proportional to $\cos \psi$ that would cause unbounded growth) must vanish.

Solving for $\dot{\beta}$:
\begin{equation}
    \dot{\beta} = \frac{3\alpha a^2}{8\omega_0}
\end{equation}

The total instantaneous frequency is $\omega(a) = \omega_0 + \dot{\beta}$:
\begin{equation}
    \omega(a) = \omega_0 + \frac{3\alpha}{8\omega_0} a^2
\end{equation}

\section{The Backbone Curve}

Squaring the instantaneous frequency:
\begin{equation}
    \omega^2 = \left( \omega_0 + \frac{3\alpha}{8\omega_0} a^2 \right)^2 = \omega_0^2 + \frac{3\alpha}{4} a^2 + \mathcal{O}(a^4)
\end{equation}

Neglecting higher-order terms (valid for moderate amplitudes), we obtain the linear **Backbone Curve**:
\begin{equation}
    \boxed{\omega^2 = \omega_0^2 + \left(\frac{3\alpha}{4}\right) a^2}
    \label{eq:backbone}
\end{equation}

\section{Validity and Robustness}

\subsection{Independence from Damping}
Note that the damping coefficient $\gamma$ does not appear in Eq. (\ref{eq:backbone}). While damping causes $a(t)$ to decay over time, the functional relationship between the instantaneous squared frequency $\omega^2(t)$ and squared amplitude $a^2(t)$ remains invariant. This allows $\alpha$ to be extracted even from heavily damped signals, provided the ridge can be tracked.

\subsection{Validity Conditions}
\begin{enumerate}
    \item \textbf{Single Mode:} The derivation assumes a single degree of freedom. Mode coupling acts as noise or distortion to this relationship.
    \item \textbf{Slow Decay:} The approximation holds when relative change in amplitude per cycle is small ($\gamma \ll \omega_0$).
    \item \textbf{Weak Nonlinearity:} The expansion assumes the frequency shift is small relative to $\omega_0$.
\end{enumerate}

\section{Conclusion}
The linear relationship between $\omega^2$ and $a^2$ provides a robust metric for identifying $\alpha$. By plotting $\omega^2(t)$ vs $a^2(t)$ extracted via Wavelet Transform, $\alpha$ can be determined from the slope $m$:
\begin{equation}
    \alpha = \frac{4}{3} m
\end{equation}

\end{document}
